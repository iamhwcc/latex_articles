\documentclass[12pt]{article}
\usepackage[GBK]{ctex}
\usepackage{amsmath}

\begin{document}

\section*{Elasticsearch 聚合类型习题}

\subsection*{一、度量聚集(Metric Aggregations)}

\textbf{题目 1: 最小值与最大值}  

在一个索引中,字段 `temperature` 表示某城市的每日温度。编写 Elasticsearch 查询,使用度量聚集分别计算最高温度和最低温度。

\begin{verbatim}
POST index/_search
{
    "query":{
        "match_all": {}
    },
    "size": 0,
    "aggs": {
        "max_tem": {
            "max": {
                "field": "temperature",
                "missing": 0
            }
        },
        "min_tem": {
            "min": {
                "field": "temperature",
                "missing": 0
            }
        }
    }
}
\end{verbatim}

\textbf{题目 2: 唯一值统计与百分比计算}  

在一个索引中,字段 `user\_id` 表示用户 ID,`spend` 表示用户的消费金额。  
(1) 使用 `cardinality` 度量聚集计算唯一用户数量;  
(2) 使用百分比分位数聚集计算消费金额的 50\% 和 95\% 分位数。

\begin{verbatim}
POST index/_search
{
    "query":{
        "match_all": {}
    },
    "size": 0,
    "aggs": {
        "cardinality_user_id": {
            "cardinality": {
                "field": "user_id",
                "missing": 0
            }
        },
        "percentiles_spend": {
            "percentiles": {
                "field": "spend",
                "missing": 0
            }
        }
    }
}
\end{verbatim}


\textbf{题目 3: 平均值计算}  

假设我们有一个包含销售记录的索引,其中每条记录包含字段 `price` 表示商品价格,`quantity` 表示商品数量。编写 Elasticsearch 查询,使用度量聚集计算商品价格的平均值。

\begin{verbatim}
POST index/_search
{
    "query": {
        "match_all": {}
    },
    "size": 0,
    "aggs": {
        "avg_price": {
            "avg": {
                "field": "price",
                "missing": 0
            }
        }
    }
}
\end{verbatim}

\textbf{题目 4: 唯一值统计}  

在一个索引中,字段 `user\_id` 表示用户标识,可能存在重复值。编写 Elasticsearch 查询,使用度量聚集计算唯一用户数量。

\begin{verbatim}
POST index/_search
{
    "query": {
        "match_all": {}
    },
    "size": 0,
    "aggs": {
        "cardinality_id": {
            "cardinality": {
                "field": "user_id"
            }
        }
    }
}
\end{verbatim}

\subsection*{二、桶聚集(Bucket Aggregations)}

\textbf{题目 1: 按术语分组统计}  

一个索引中包含字段 `category` 和 `sales`,分别表示商品类别和销量。编写 Elasticsearch 查询,使用 `terms` 桶聚集统计每个类别的总销量,输出最大的5个。

\begin{verbatim}
POST index/_search
{
    "query": {
        "match_all": {}
    },
    "size": 0,
    "aggs": {
        "group_by_category": {
            "terms": {
                "field": "category",
                "size": 5,
                "order": {
                    "sum_price": desc
                }
            },
            "aggs": {
                "sum_price": {
                    "sum": {
                        "field": sales
                    }
                }
            }
        }
    }
}
\end{verbatim}


\textbf{题目 2: 时间分桶与范围分桶}  

(1) 一个索引中,字段 `order\_date` 表示订单日期,`revenue` 表示订单收入。编写 Elasticsearch 查询,使用 `date\_histogram` 桶聚集按季度分组统计每季度的收入; \\
\begin{verbatim}
POST index/_search
{
    "query": {
        "match_all": {}
    },
    "size": 0,
    "aggs": {
        "date": {
            "date_histogram": {
                "field": "order_date",
                "calendar_interval": "quarter"
            },
            "aggs": {
                "sum_revenue": {
                    "sum": {
                        "field": "revenue",
                        "missing": 0
                    }
                }
            }
        }        
    }
}
\end{verbatim}

(2) 字段 `age` 表示用户年龄,编写查询使用 `range` 桶聚集按年龄范围(如 18-25,26-35)统计用户数量。 \\ 

\begin{verbatim}
POST index/_search
{
    "query": {
        "match_all": {}
    },
    "size": 0,
    "aggs": {
        "agg_range": {
            "range": {
                "field": "age",
                "ranges": [
                    {"from": 18, "to": 25},
                    {"from": 26, "to": 35}
                ]
            },
            "aggs": {
                "user_count": {
                    "value_count": {
                        "field": "age"
                    }
                }
            }
        }
    }
}
\end{verbatim}


\subsection*{三、管道聚集(Pipeline Aggregations)}

\textbf{题目 1: 最大桶与平均桶}  

在一个索引中,字段 `sales\_date` 表示销售日期,`sales` 表示每日销量。\\
(1) 使用 `max\_bucket` 管道聚集计算销量最高的一周;\\
\begin{verbatim}
POST index/_search
{
    "query": {
        "match_all": {}
    },
    "size": 0,
    "aggs": {
        "sum_month": {
            "date_histogram": {
                "field": "age",
                "fixed_interval": "7d"
            },
            "aggs": {
                "sales_sum": {
                    "sum": {
                        "field": "sales"
                    }
                }
            }
        },
        "max_bkt": {
            "max_bucket": {
                "buckets_path": "sum_month>sales_sum"
            }
        }
    }
}
\end{verbatim}
(2) 使用 `avg\_bucket` 管道聚集计算每周平均销量。\\
\begin{verbatim}
POST index/_search
{
    "query": {
        "match_all": {}
    },
    "size": 0,
    "aggs": {
        "sum_month": {
            "date_histogram": {
                "field": "age",
                "fixed_interval": "7d"
            },
            "aggs": {
                "sales_sum": {
                    "sum": {
                        "field": "sales"
                    }
                }
            }
        },
        "max_bkt": {
            "avg_bucket": {
                "buckets_path": "sum_month>sales_sum"
            }
        }
    }
}
\end{verbatim}

\end{document}
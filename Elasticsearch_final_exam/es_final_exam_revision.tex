\documentclass[]{exam}
\usepackage[GBK]{ctex}
\usepackage{graphicx} % Required for inserting images
\usepackage{amsmath}
\usepackage{xcolor} % 加载颜色包

% ----- choice format properties
\makeatletter
\renewenvironment{choices}%
    {\list{\choicelabel}%
       {\usecounter{choice}\def\makelabel##1{\hss\llap{##1}}%
         \setlength{\leftmargin}{15pt}%
         \def\choice{%
           \if@correctchoice
             \color@endgroup
             \endgroup
           \fi
           \item
           \do@choice@pageinfo
         } % choice
         \def\CorrectChoice{%
           \if@correctchoice
             \color@endgroup
             \endgroup
           \fi
           \ifprintanswers
             \ifhmode \unskip\unskip\unvbox\voidb@x \fi
             \begingroup \color@begingroup \@correctchoicetrue
             \CorrectChoice@Emphasis
           \fi
           \item
           \do@choice@pageinfo
         } % CorrectChoice
         \let\correctchoice\CorrectChoice
         \labelwidth\leftmargin\advance\labelwidth-\labelsep
         \topsep=0pt
         \partopsep=0pt
         \choiceshook
       }%
    }%
  {\if@correctchoice \color@endgroup \endgroup \fi \endlist}
\makeatother
% ----- choice format properties

\title{Elasticsearch Final Revision}
\author{Weicong HUANG, stalwarthuang@outlook.com}
\date{December 2024}

\begin{document}
\maketitle

\vspace{-2.5em} % 调整声明与标题之间的间距
\begin{center}
\textit
{
These two tests are class tests for the course 《Big Data Processing Techniques - Elasticsearch》at SCAU. They are only for personal research and study exchanges!
}
\end{center}
\vspace{0.8em} 

\section{Class Test 1}

\begin{questions}
\question
关于 Elasticsearch 的描述中, 不正确的是
\begin{choices}
    \choice Elasticsearch 是解决海量数据全文检索的不二之选
    \choice \textcolor{red}{Elasticsearch 只能为结构化数据提供搜索和分析服务}
    \choice ES 是一个基于 Java 语言开发的, 基于 Lucene 的开源分布式搜索引擎
    \choice 只要是用到搜索的场景, ES 几乎都可以说是最好的选择
\end{choices}
\textit{\textcolor{red}{Elasticsearch 为所有类型的数据提供近乎实时的搜索和分析}}

\question
ES 的典型应用场景不包括
\begin{choices}
    \choice 在线实时日志分析
    \choice 物联网数据监控
    \choice \textcolor{red}{事务场景}
    \choice 文献检索和文献计量
\end{choices}

\question
访问 ES 集群服务器的端口地址默认是
\begin{choices}
    \choice \textcolor{red}{9200}
    \choice 9220
    \choice 9020
    \choice 9300
\end{choices}

\question
ES 集群中, 节点之间相互通信的默认端口号是
\begin{choices}
    \choice 9200
    \choice \textcolor{red}{9300}
    \choice 9100
    \choice 9230
\end{choices}

\question
搜索引擎中的反向索引(倒排索引)是指
\begin{choices}
    \choice 通过文章找词
    \choice 通过文章找文章
    \choice 通过词找词
    \choice \textcolor{red}{通过词找文章}
\end{choices}

\question
关于 ES 集群中索引和分片的描述, 不正确的是
\begin{choices}
    \choice 索引是由分片(shards)组成, 并且分片可以有副本
    \choice 分片的种类包括主分片和副本分片
    \choice 分片可以提高服务的高可用性
    \choice \textcolor{red}{主分片和其副本分片可以同时存在于同一个节点上}
\end{choices}

\question
ES 集群节点的角色不包括
\begin{choices}
    \choice 主节点(master node)
    \choice 数据节点(data node)
    \choice 预处理节点(ingest node)
    \choice \textcolor{red}{从节点(slave node)}
\end{choices}

\question
关于索引的说法中, 不正确的是
\begin{choices}
    \choice 索引名称必须用小写字符
    \choice 索引的主分片数量定义后不能修改
    \choice 索引的副本分片数量定义后可修改
    \choice \textcolor{red}{从节点当向索引中添加数据的字段是原先未定义的, 数据不可以添加}
\end{choices}
\textit{\textcolor{red}{当向索引中添加数据的字段是原先未定义的, 数据依然可以被成功添加。ES拥有动态映射机制, 会根据数据的内容自动识别对应的字段类型。}}

\question
关于索引中文档主键的描述, 正确的是
\begin{choices}
    \choice \textcolor{red}{添加文档数据时如果没有指定主键, 则系统会生成一个不重复的字符串作为主键}
    \choice ES 的索引中的文档, 可以没有主键
    \choice 写入索引数据时, 如果文档主键已存在则会报错
\end{choices}

\question
关于索引的健康状态的描述, 正确的是
\begin{choices}
    \choice 蓝色表示所有分片(主、副本)都可用
    \choice 绿色表示至少有一个副本不可用, 但所有主分片都可用
    \choice 橙色表示至少有一个副本不可用, 但所有主分片都可用
    \choice \textcolor{red}{红色表示至少有一个主分片不可用, 数据不完整}
\end{choices}
\textit{\textcolor{red}{
- 绿色:所有分片都可用 \\
- 黄色:至少有一个副本不可用, 但是所有主分片都可用, 此时集群能提供完整的读写服务, 但是可用性较低。\\
- 红色:至少有一个主分片不可用, 数据不完整。此时集群无法提供完整的读写服务。集群不可用}}

\question
当一个索引数据量太大时, 继续写入可能导致分片数据量过大, 查询时会因内存不足引起集群崩溃;为避免所有数据都写入同一个索引, 可以使用\_\_\_\_技术。\\
该技术需要配合索引别名一起使用, 可实现把原先写入一个索引的数据自动分发到多个索引中。
\begin{choices}
    \choice \textcolor{red}{滚动索引}
    \choice 索引模板
    \choice 动态映射
    \choice 字段复制
\end{choices}

\question
文本分析的描述中, 错误的是:
\begin{choices}
    \choice 文本分析器包含:大于等于零个字符过滤器、一个分词器、大于等于零个分词过滤器
    \choice 文档入库时, 任何 text 类型字段都会进行文本分析
    \choice 检索已入库文档时, 对于查询的字段会进行文本分析
    \choice \textcolor{red}{文本分析器就是文本分词器}
\end{choices}

\question
标准分析器 standard 包含
\begin{choices}
    \choice \textcolor{red}{标准分词器和一个小写分词过滤器}
    \choice 简单分词器和一个小写分词过滤器
    \choice 只包含标准分词器, 没有其他过滤器
    \choice 标准分词器和一个标点符号去除器
\end{choices}

\question
IK 中文分词器的描述中, 不正确的是:
\begin{choices}
    \choice 分词两种:ik\_smart、ik\_max\_word
    \choice 全文检索时文本分析使用 ik\_smart 较为常见
    \choice 索引时文本分析使用 ik\_max\_word 更加合适
    \choice \textcolor{red}{同一中文句子使用 ik\_smart 分词后得到的词数量一般比ik\_max\_word 分词后得到的词数量多}
\end{choices}

\question
查询索引 xx 的映射的命令是:
\begin{choices}
    \choice get xx/\_search
    \choice get xx/map
    \choice \textcolor{red}{get xx/\_mapping}
    \choice get xx/mapping
\end{choices}

\question
查看集群所有索引的命令是:
\begin{choices}
    \choice \textcolor{red}{get \_cat/indices}
    \choice get cat/indices
    \choice get all/indices
    \choice get indices
\end{choices}

\question
使用 Kibana 时, 浏览器端的默认端口号是:
\begin{choices}
    \choice 6501
    \choice 5600
    \choice \textcolor{red}{5601}
    \choice 5061
\end{choices}

\question 
\begin{verbatim}
POST mydata/_search
{
  "query": {
    "match_all": {}
  },
  "size": 10,
  "from": 0
}
\end{verbatim}
关于语句的描述中, 错误的是:
\begin{choices}
    \choice 索引名称是 mydata
    \choice \_search 表示查询
    \choice match\_all 表示查询所有文档
    \choice \textcolor{red}{size 为 10 表示查询前 10 个文档}
\end{choices}
\textit{\textcolor{red}{size表示分页大小, "size":10表示返回10个文档, 而并非前10个, 结合了"from":0才是返回前10个}}

\question
\begin{verbatim}
POST _analyze
{
  "analyzer": "ik_smart",
  "text": "数据科学与大数据专业"
}
\end{verbatim}
字符串 “数据科学与大数据专业” 使用 ik\_smart 分词器得到的分词结果是:
\begin{choices}
    \choice 数据, 科学, 与, 大, 专业
    \choice \textcolor{red}{数据, 科学, 与, 大, 数据, 专业}
    \choice 数据科学, 大数据, 专业
    \choice 数据, 科学, 专业
\end{choices}
\textit{\textcolor{red}{
--未设置停用词, 所以包含“与”\\
--大数据可以拆分为“ 大”, “数据”\\
数据\\
科学\\
与\\
大\\
数据\\
专业}}

\question
\begin{verbatim}
POST test-3-2-1/_search
{
  "query": {
    "term": {
      "name.keyword": {
        "value": "张三"
      }
    }
  }
}
\end{verbatim}
查询语句的描述中, 错误的是:
\begin{choices}
    \choice term 表示术语查询
    \choice name.keyword 此处不可以改为 name
    \choice \textcolor{red}{该语句表示查询姓名为 “张三” 的文档, 其中包括 “张三丰” 的文档}
    \choice 该语句是以一个精准查询, 而不是模糊匹配
\end{choices}
\textit{\textcolor{red}{term精准查询"张三", 不包括张三丰}}

\end{questions}

\section{Class Test 2}

\begin{questions}
    
\question
关于ES特点的描述, 错误的是: 
\begin{choices}
    \choice 基于Java语言开发
    \choice 基于Lucene框架
    \choice 原生支持分布式
    \choice \textcolor{red}{只支持TB级数据量}
\end{choices}
\textit{\textcolor{red}{可支持PB级数据量}}

\question
ES集群的节点有多种类型,其中不包括:
\begin{choices}
    \choice master node
    \choice data node
    \choice \textcolor{red}{input node}
    \choice coordinating node
\end{choices}
\textit{\textcolor{red}{数据接入节点: ingest node}}

\question
关于分片的策略,错误的描述是:
\begin{choices}
    \choice 主分片和其副本分片不能同时存在于同一个节点上
    \choice 每个分片都是一个Lucene实例
    \choice ES会自动在nodes上做分片均衡shard rebalance
    \choice \textcolor{red}{完全相同的副本可以同时存在于同一个节点上}
\end{choices}
\textit{\textcolor{red}{完全相同的副本“不能”不能同时存在于同一个节点上}}

\question
修改一个索引文档内容时,使用的REST方法是
\begin{choices}
    \choice \textcolor{red}{POST}
    \choice {UPDATE}
    \choice {REPLACE}
    \choice {UPSERT}
\end{choices}
\textit{\textcolor{red}{REST方法包括:PUT,GET,POST,DELETE,HEAD。}}

\question
向索引输入一条文档数据时,其REST方法是
\begin{choices}
    \choice GET
    \choice \textcolor{red}{POST}
    \choice {DELETE}
    \choice PUTS
\end{choices}

\question
索引数据批量录入时的指令方法是
\begin{choices}
    \choice BAT
    \choice BATS
    \choice \textcolor{red}{BULK}
    \choice {INPUT}
\end{choices}

\question
定义一个字段的类型是不经过文本分析处理的关键字类型是
\begin{choices}
    \choice text类型
    \choice key类型
    \choice keytext类型
    \choice \textcolor{red}{keyword类型}
\end{choices}

\question
关于索引路由的描述中,错误的是
\begin{choices}
    \choice \textcolor{red}{路由的计算公式与索引主分片数量无关}
    \choice 路由的计算公式与索引主分片数量有关
    \choice 路由计算公式中的\_routing,默认是文档的\_id值
    \choice 路由的本质就是计算新数据属于哪个分片
\end{choices}
\textit{\textcolor{red}{
路由计算公式:$shard\_num = \text{hash}(\_routing) \% num\_primary\_shards$ \\
其中 \textit{\_routing} 与主键有关,\textit{num\_primary\_shards} 表示主分片的数量。
}}

\question
在dev-tools中,查询ES集群所有索引的命令正确的是
\begin{choices}
    \choice GET \_cat/allindex
    \choice \textcolor{red}{GET \_cat/indices}
    \choice GET \_cat/index
    \choice GET \_cat/allindices
\end{choices}

\question
搜索数据时,术语查询是指\_\_\_查询
\begin{choices}
    \choice match查询
    \choice match\_all
    \choice \textcolor{red}{term}
    \choice regexp
\end{choices}

\question
kibana的默认web ui端口是
\begin{choices}
    \choice 6500
    \choice 6501
    \choice 5600
    \choice \textcolor{red}{5601}
\end{choices}

\question
在管道聚集中,如果要定义一个平均管道聚集,则其关键词是
\begin{choices}
    \choice sum\_bucket
    \choice min\_bucket
    \choice max\_bucket
    \choice \textcolor{red}{avg\_bucket}
\end{choices}

\question
在父子关联索引数据设计方法中,关于join字段方法的描述中,错误的是
\begin{choices}
    \choice 父子关系数据写入同一个索引
    \choice \textcolor{red}{父子关系数据写入不同的两个索引}
    \choice 父文档和子文档分开输入
    \choice 每个子文档都是独立的文档
\end{choices}

\question
索引的健康状态显示黄色,描述错误的是
\begin{choices}
    \choice \textcolor{red}{有主分片没有分配,索引不可用}
    \choice 主分片都已分配,但有副本分片没分配,索引可用
    \choice 所有主分片都已分配,索引可用
    \choice 黄色表示索引可用。当所有主分片和副本分片都分配后,索引状态变为绿色
\end{choices}
\textit{\textcolor{red}{黄色表示,亚健康,集群可用;\\
→所有主分片都已分配,有副本分片没有分配}}

\question
获取ES集群健康情况的操作指令是
\begin{choices}
    \choice GET \_cat/index
    \choice GET \_cat/nodes
    \choice \textcolor{red}{GET \_cat/health}
    \choice GET \_cat/master
\end{choices}

\question
如果需要监控主机性能指标数据,则在Beats采集家族中选用的工具是
\begin{choices}
    \choice filebeat
    \choice \textcolor{red}{metricbeat}
    \choice packetbeat
    \choice Heartbeat
\end{choices}


\question 判断一个索引test\_3是否存在时, 使用语句HEAD test-3, 如果返回结果是200, 则表示索引存在 \hfill \textbf{(T)}

\question 新建、索引和删除请求都是“写”操作, 必须在主分片上面完成之后才能被复制到相关的副本分片 \hfill \textbf{(T)}

\question 根据文档\_id检索文档时,可以从主分片或者从其它任意副本分片检索文档; \hfill \textbf{(T)}

\question 使用标准分析器处理文本时,不进行字母的小写转换 \hfill \textbf{(F)} \\
\textit{\textcolor{red}{标准分析器standard analyzer会根据Unicode文本分割算法定义的单词边界将文本划分为术语。它删除了大多数标点符号并将单词转为小写,并支持删除停用词}}

\question 为某个字段指定ignore\_malformed为false表示:即使该字段输入数据类型不匹配,不影响其他字段的输入 \hfill \textbf{(F)} \\
\textit{\textcolor{red}{"ignore\_malformed":true 才能保证输入数据类型不匹配时不影响其他字段输入}}

\question 向索引中写入一个文档包含事先未定义的字段内容时,写入操作不成功;\hfill \textbf{(F)} \\
\textit{\textcolor{red}{ES有动态映射功能}}

\question 管道聚集与桶聚集无关 \hfill \textbf{(F)} \\
\textit{\textcolor{red}{管道聚集必须有一个输入才能正常工作,这个输入是桶聚集生成的结果,故管道聚集与桶聚集有关(管道聚集第一步就是桶聚集)}}

\question 采用Logstash采集数据时,不添加额外的信息,数据原封不动输出下游 \hfill \textbf{(F)} \\
\textit{\textcolor{red}{即使不添加额外信息,Logstash也会自动添加一些元数据信息,比如@timestamp @version等等}}

\question beats在数据收集层面上并不进行过于复杂的数据处理,只是将数据简单的组织并上报给上游系统 \hfill \textbf{(T)}

\section{Aggregations}
1. 既计算了总共的sales总和,也计算了每个桶内的sales总和。
\begin{verbatim}
POST index/_search
{
  "query": {
    "match_all": {}
  },
  "size": 0,
  "aggs": {
    "total_sales": {
      "sum": {
        "field": "sales"
      }
    },
    "group_by_category": {
      "terms": {
        "field": "category",
        "size": 5,
        "order": {
          "total_sales": "desc"
        }
      },
      "aggs": {
        "total_sales": {
          "sum": {
            "field": "sales"
          }
        }
      }
    }
  }
}
\end{verbatim}

\section{Common commands}
\begin{enumerate}
    \item 查看所有索引 \\
    GET \_cat/indices
    \item 查看集群健康状况 \\
    GET \_cat/health
    \item 查看主节点 \\ 
    GET \_cat/master
    \item 查看所有节点 \\ 
    GET \_cat/nodes
    \item 查看索引映射 \\
    GET index\_name/\_mapping
    \item 查看索引所有数据 \\
    GET index\_name/\_search
    \item 检测索引是否存在 \\
    HEAD index\_name (200: 存在;404: 不存在)
    \item 查看设置的索引别名 \\
    GET \_aliases \\
    {https://localhost:9200/\_cat/\_aliases}
\end{enumerate}













\end{questions}

\end{document}
